\subsection{Objectives}

This work intends to study \glspl{cpe} provided to customers of one Brazillian \gls{isp}. The \gls{isp} was chosen based on its national relevance and on the ability of the researcher to run the proposed actions with the permission of multiple customers on their environment. Due to concerns regarding a direct reference to the provider, the \gls{isp} name will not be disclosed.

The main objective of the research is to check if the devices are being properly configured and provide a desired level of security. To better understand the impact of using a standard for authentication and encryption over another on a \gls{cpe}, the protocols and their security properties are analyzed and known issues and vulnerabilities are studied. If some issue is unveiled, the risks and possible mitigations are discussed.

Tests are performed on the management interfaces of the \glspl{cpe} to ensure that they are robust enough against attacks that may compromise the network, the devices connected to it, or the device itself.

The \gls{isp} servers that may impact on the customer privacy and security are analyzed and the connection from the residential gateway to them is inspected. This work specially aims to study the servers that handle the telephony service with the \gls{sip} and the ones that perform remote management of the \glspl{cpe} with the \gls{cwmp}.

It is also investigated if a customer is able to replace the \gls{isp} \gls{cpe} with a piece of commercial equipment, taking the responsibility of maintaining the device from the \gls{isp} and leaving to itself. The process of doing so is detailed and the drawbacks are discussed.

Finally, recommendations based on the findings of this research and on the background knowledge are given. Action points that the customer may perform to improve its security are explored for the different pieces of equipment.
