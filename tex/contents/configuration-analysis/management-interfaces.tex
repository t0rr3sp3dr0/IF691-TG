\subsection{Management Interfaces}

The \glspl{cpe} analyzed have two users registered, admin and support. All management interfaces can be accessed using these users, but permissions may differ.

Both users share the same password, that is printed on each device label. Although patterns have been recognized in the password of all \gls{cpe}s, as presented in Table 10, none of them were identified to be derived from other information.

\begin{table}[h]
    \makebox[\linewidth]{
        \begin{tabular}{c|c}
            \thead{\gls{cpe} Identifier} & \thead{Password Pattern} \\
            \hline
            \gls{cpe}-0 & \texttt{[0-9A-Za-z]\{8\}} \\
            \gls{cpe}-1 & \texttt{[0-9A-Za-z]\{8\}} \\
            \gls{cpe}-2 & \texttt{[0-9a-z]\{8\}} \\
            \gls{cpe}-3 & \texttt{[0-9a-z]\{8\}} \\
            \gls{cpe}-4 & \texttt{[0-9a-f]\{8\}} \\
            \gls{cpe}-5 & \texttt{[0-9a-z]\{8\}} \\
            \gls{cpe}-6 & \texttt{[0-9a-f]\{8\}} \\
            \gls{cpe}-7 & \texttt{[0-9a-f]\{8\}} \\
        \end{tabular}
    }
    \caption{Management Interface Password Pattern of the \gls{cpe}s}
    \label{table:cpes_mgtiface_pattern}
\end{table}

An online brute-force attack was performed on the \gls{http} Management Interfaces of the \glspl{cpe} by writing a C program that opens a socket with the server and sends requests as fast as possible, but the results indicate that it is not feasible to exploit it with this technique. The \gls{http} server is slow and seems to only process one request at a time, the performance decreased significantly when parallelizing the attack due to the sequential processing of the \gls{http} server.

Although, if some way of extracting the hashed passwords from the devices is discovered, most of the passwords could be cracked in a reasonable amount of time giving the pattern constraints. The kind of vulnerability that allows the hash extraction is not common, but it has been on some \glspl{crg} in recent years.

\FloatBarrier
