\paragraph{Security}

A compromise was made on the \gls{tkip} design to make possible its use on legacy \gls{wep} hardware. The forgery attacks were mitigated with the introduction of the Michael algorithm, but, due to computing power constraints, while blocking the forged packets it would cause a denial of service on the network \cite{ieee_80211_2020}.

As \gls{tkip} just encapsulates the \gls{wep} algorithm, it still relies on the security of the \gls{rc4} \gls{prng}. It was found that the keystream generated by \gls{rc4} is biased towards certain sequences and it made practical attacks against \gls{wpa}-\gls{tkip} networks within an hour \cite{rc4nomore}. The attacker would establish a \gls{tcp} connection with some victim on the network and would repeatedly send identical packets particularly sized with a well-known content over the connection. Then the wireless traffic was captured and filtered to only what would likely be an attacker's packet. Ciphertext statistics were extracted and plaintext likelihoods calculated using a combination of the \gls{fm} and \gls{absab} biases. Finally, the \gls{mic} key is derived from one of the candidates with the correct \gls{icv}, allowing any other packet of the victim to be fully decrypted.
