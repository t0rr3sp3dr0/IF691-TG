\subsection{IEEE 802.11 and Wi-Fi}

\gls{ieee} 802.11 is a working group, part of the \gls{ieee} \gls{lmsc}, that develops and maintains networking standards and recommended practices for \gls{wlan} \cite{about_ieee_p80211}. The first version of its standard was released in 1997 and operated at a 2.4 \gls{G}\gls{Hz} frequency, allowing data to be transmitted up to 2 \gls{M}\gls{b}\gls{ps}.

The original standard was quickly superseded by the \gls{ieee} 802.11b amendment. Published in 1999, it increased the maximum transfer rate of the standard to 11 \gls{M}\gls{b}\gls{ps} while staying in the 2.4 \gls{G}\gls{Hz} band.

In the same year, companies joined forces together to certify interoperability of products implementing the \gls{ieee} 802.11 standard and to promote it as the global \gls{wlan} standard across all market segments. The \gls{weca} was born with that mission \cite{weca_mission}.

For months, \gls{weca} worked closely with several vendors of \gls{ieee} 802.11b equipment to develop an interoperability testbed for the standard. The work was complete in February 2000 and \gls{weca} members were able to submit their products against the test matrix. Upon success, the product would receive the \gls{wifi} CERTIFIED\textsuperscript{TM} seal and the company would be allowed to use the \gls{wifi}\textsuperscript{TM} logo on the device’s advertising and packaging. The seal indicates that the product "met industry-agreed standards for interoperability, security, and a range of application specific protocols" and works with any other devices that also bear the symbol \cite{wifi_certification}. In 2002, \gls{weca} changed its name to \gls{wifi} Alliance.

\begin{table}[h]
    \makebox[\linewidth]{
        \begin{tabular}{c|c}
            \thead{\gls{ieee} 802.11 Amendment} & \thead{\gls{wifi} Generation} \\
            \hline
            802.11b & \gls{wifi} 1 \\
            802.11a & \gls{wifi} 2 \\
            802.11g & \gls{wifi} 3 \\
            802.11n & \gls{wifi} 4 \\
            802.11ac & \gls{wifi} 5 \\
            802.11ax & \gls{wifi} 6 \\
        \end{tabular}
    }
    \caption{\gls{ieee} 802.11 Amendment to \gls{wifi} Generation mapping}
    \label{table:map_80211_wifi}
\end{table}

In 2018, the \gls{wifi} Alliance introduced a new designation to identify the \gls{wifi} generations \cite{wifi6_introduction}, moving away from the traditional \gls{ieee} 802.11 amendment names and adopting a simpler numerical sequence scheme as shown in Table \ref{table:map_80211_wifi}.

\FloatBarrier
